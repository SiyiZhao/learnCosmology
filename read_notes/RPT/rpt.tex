\documentclass{ctexart}

%style.tex
\usepackage[top=30mm, bottom=30mm, textwidth=120mm, left=25mm, right=25mm, outer=40mm, marginparsep=8mm, headsep=20pt]{geometry}
\usepackage{xcolor}
\usepackage{soul} % for highlight

\usepackage{amsmath}
\usepackage{amsthm}
\usepackage{physics}

\usepackage[colorlinks = true,
            linkcolor = blue,
            urlcolor  = blue,
            citecolor = blue,
            anchorcolor = blue]{hyperref}
\usepackage{orcidlink} 

\definecolor{RoyalBlue}{rgb}{0.25,0.41,0.88}
% \newenvironment{discuss}
% {% 这是环境开始时的代码
%   \begin{center}
%   \begin{tabular}{|c|c|}
%   \hline
% }
% {% 这是环境结束时的代码
%   \hline
%   \end{tabular}
%   \end{center}
% }
\newcommand{\helper}[1]{\textcolor{RoyalBlue}{\textbf{#1}}}
\newcommand{\discuss}[1]{\textcolor{RoyalBlue}{\textit{讨论问题:}\textbf{#1}}}

% \newtheoremstyle{SettingTheorem} % name
%     {3pt} % Space above
%     {3pt} % Space below
%     {\itshape} % Body font
%     {} % Indent amount
%     {\bfseries} % Theorem head font
%     {.} % Punctuation after theorem head
%     {.5em} % Space after theorem head
%     {} % Theorem head spec (can be left empty, meaning ‘normal’)

\theoremstyle{SettingTheorem}
\newtheorem{block}{Block}

\newcommand{\refeq}[1]{Eq.~(\ref{eq:#1})}

\DeclareMathOperator{\Ham}{H}

\newcommand{\adot}{\dot{a}}





\begin{document}

\title{Relativistic Perturbation Theory}
\author{Siyi Zhao (赵思逸)\,\orcidlink{0009-0001-4492-5158}}
\date{\today}

\maketitle
\tableofcontents

\section{Metric Perturbation}

\begin{equation}
    \mathrm{d} s^2=a^2(\eta)\left[-(1+2 A) \mathrm{d} \eta^2+2 B_i \mathrm{~d} x^i \mathrm{~d} \eta+\left(\delta_{i j}+2 E_{i j}\right) \mathrm{d} x^i \mathrm{~d} x^j\right]
\end{equation}

\subsection{SVT Decomposition}

\begin{equation}
    B_i=\underbrace{\partial_i B}_{\text {scalar }}+\underbrace{\hat{B}_i}_{\text {vector }}
\end{equation}

\begin{equation}
    E_{i j}=\underbrace{C \delta_{i j}+\partial_{\langle i} \partial_{j\rangle} E}_{\text {scalar }}+\underbrace{\partial_{(i} \hat{E}_{j)}}_{\text {vector }}+\underbrace{\hat{E}_{i j}}_{\text {tensor }},
\end{equation}
where   
\begin{equation}
    \begin{aligned}
    \partial_{\langle i} \partial_{j\rangle} E & \equiv\left(\partial_i \partial_j-\frac{1}{3} \delta_{i j} \nabla^2\right) E \\
    \partial_{(i} \hat{E}_{j)} & \equiv \frac{1}{2}\left(\partial_i \hat{E}_j+\partial_j \hat{E}_i\right) .
    \end{aligned}
\end{equation}

\section{Coordinate Transformations}

\begin{equation}
    x^\mu(q) \mapsto \tilde{x}^\mu(q) \equiv x^\mu(q)+\xi^\mu(q), \quad 
    \text{where} \quad 
    \begin{aligned}
        & \xi^0 \equiv T, \\
        & \xi^i \equiv L^i=\partial^i L+\hat{L}^i.
    \end{aligned}
\end{equation}

\section{Gauge Transformations of Metric Perturbations}

Assume $A$, $B_i$ and $E_{ij}$ are in order $\xi$.

\begin{example}{The transformation of $B_i$}
    \begin{equation}
        \begin{aligned}
            a^2(\eta) B_i = g_{0i}(x) &= \frac{\partial \tilde{x}^\alpha}{\partial \eta} \frac{\partial \tilde{x}^\beta}{\partial x^i} \tilde{g}_{\alpha \beta} (\tilde{x}) \\ 
            &= a^2(\eta+T) \qty[\underbrace{-\partial_i T}_{\text {00-term }} + \underbrace{\delta^j_i  \tilde{B}_{j}}_{\text {0j-term }} + \underbrace{L^{j \prime} \delta^k_i \delta_{jk}}_{\text{jk-term}}] + \Ocal(\xi^2) 
        \end{aligned}
    \end{equation}
    where j0-term is $0$ since both $\frac{\partial \tilde{x}^j}{\partial \eta}$ and $\frac{\partial \tilde{\eta}}{\partial x^i}$ are perturbations.
    \begin{equation}
        B_i = (1 + 2 \Hcal T) \qty[-\partial_i T + \tilde{B}_{i} + L^{i \prime}] + \Ocal(\xi^2) 
    \end{equation}
    So 
    \begin{equation}
        B_i \rightarrow \tilde{B}_{i} = B_i + \partial_i T - L^{i\prime} + \Ocal(\xi^2) 
    \end{equation}
\end{example}

\begin{example}{The transformation of $E_{ij}$}
    \begin{equation}
        \begin{aligned}
            g_{ij}(x) &= \frac{\partial \tilde{x}^\alpha}{\partial x^i} \frac{\partial \tilde{x}^\beta}{\partial x^j} \tilde{g}_{\alpha \beta} (\tilde{x}) 
        \end{aligned}
    \end{equation}
    
\end{example}
\begin{exercise}{6.1}
    SVT    

\end{exercise}

\section{Energy-momentum Tensor}
\begin{equation}
    \begin{aligned}
    T_0^0 & \equiv-(\bar{\rho}+\delta \rho) \\
    T_i^0 & \equiv(\bar{\rho}+\bar{P}) v_i=-T_0^i \\
    T_j^i & \equiv(\bar{P}+\delta P) \delta_j^i+\Pi_j^i, \quad \Pi_i^i \equiv 0
    \end{aligned}
\end{equation}

\subsection{The Coordinate Transformations of Energy-momentum Tensor}

\begin{equation}
    T^\mu{ }_\nu(x)=\frac{\partial x^\mu}{\partial \tilde{x}^\alpha} \frac{\partial \tilde{x}^\beta}{\partial x^\nu} \tilde{T}_\beta^\alpha(\tilde{x})
\end{equation}

The Jacobi
\begin{equation}
    \frac{\partial \tilde{x}^\alpha}{\partial x^\mu}=\left(\begin{array}{cc}
    \partial \tilde{\eta} / \partial \eta & \partial \tilde{\eta} / \partial x^i \\
    \partial \tilde{x}^i / \partial \eta & \partial \tilde{x}^i / \partial x^j
    \end{array}\right)=\left(\begin{array}{cc}
    1+T^{\prime} & \partial_i T \\
    L^{i \prime} & \delta_j^i+\partial_j L^i
    \end{array}\right)
\end{equation}

The inverse matrix of Jacobi should be delivered with perturbation theory 
\begin{equation}
    \frac{\partial x^\mu}{\partial \tilde{x}^\alpha}=\left(\begin{array}{cc}
    \partial \eta / \partial \tilde{\eta} & \partial \eta / \partial \tilde{x}^i \\
    \partial x^i / \partial \tilde{\eta} & \partial x^i / \partial \tilde{x}^j
    \end{array}\right)=\left(\begin{array}{cc}
    1-T^{\prime} & -\partial_i T \\
    -L^{i \prime} & \delta_j^i-\partial_j L^i
    \end{array}\right)
\end{equation}

\begin{proof}
    $T$ 和 $L^i$ 是小量。 用 $a \sim b_i \sim c_i \sim d_{ij} \sim \epsilon$ 代替。
    \begin{equation}
        \left(\begin{array}{cc}
            1+ a & b_i \\
            c_i & \delta_j^i+ d_{ij}
            \end{array}\right) 
        \left(\begin{array}{cc}
            1- a & -b_i \\
            -c_i & \delta_j^i- d_{ij}
            \end{array}\right)  = 
        \left(\begin{array}{cc}
            1 + \mathcal{O} (\epsilon^2) & \mathcal{O} (\epsilon^2) \\
            \mathcal{O} (\epsilon^2) & \delta_j^i + \mathcal{O} (\epsilon^2)
            \end{array}\right) 
    \end{equation}
\end{proof}

\end{document}


% % \bibliographystyle{apsrev4-2}
% % \bibliography{refs}

