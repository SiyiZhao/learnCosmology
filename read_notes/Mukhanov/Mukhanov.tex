\documentclass{ctexart}

%style.tex
\usepackage[top=30mm, bottom=30mm, textwidth=120mm, left=25mm, right=25mm, outer=40mm, marginparsep=8mm, headsep=20pt]{geometry}
\usepackage{xcolor}
\usepackage{soul} % for highlight

\usepackage{amsmath}
\usepackage{amsthm}
\usepackage{physics}

\usepackage[colorlinks = true,
            linkcolor = blue,
            urlcolor  = blue,
            citecolor = blue,
            anchorcolor = blue]{hyperref}
\usepackage{orcidlink} 

\definecolor{RoyalBlue}{rgb}{0.25,0.41,0.88}
% \newenvironment{discuss}
% {% 这是环境开始时的代码
%   \begin{center}
%   \begin{tabular}{|c|c|}
%   \hline
% }
% {% 这是环境结束时的代码
%   \hline
%   \end{tabular}
%   \end{center}
% }
\newcommand{\helper}[1]{\textcolor{RoyalBlue}{\textbf{#1}}}
\newcommand{\discuss}[1]{\textcolor{RoyalBlue}{\textit{讨论问题:}\textbf{#1}}}

% \newtheoremstyle{SettingTheorem} % name
%     {3pt} % Space above
%     {3pt} % Space below
%     {\itshape} % Body font
%     {} % Indent amount
%     {\bfseries} % Theorem head font
%     {.} % Punctuation after theorem head
%     {.5em} % Space after theorem head
%     {} % Theorem head spec (can be left empty, meaning ‘normal’)

\theoremstyle{SettingTheorem}
\newtheorem{block}{Block}

\newcommand{\refeq}[1]{Eq.~(\ref{eq:#1})}

\DeclareMathOperator{\Ham}{H}

\newcommand{\adot}{\dot{a}}





\begin{document}

\title{Mukhanov 读书笔记}
\author{Siyi Zhao (赵思逸)\,\orcidlink{0009-0001-4492-5158}}
\date{\today}

\maketitle
\tableofcontents


\section*{Planck Units}

\begin{gather}
    r_{\rm Sch} = \lambda_{\rm Compton} \\
    \frac{G m_{\rm Pl}}{c^2} = \frac{\hbar}{m_{\rm Pl} c} \\ 
    m_{\rm Pl} = {\qty(\frac{\hbar c}{G})}^2
\end{gather}



\section{第4章~极早期宇宙}

\subsection{规范不变性}
规定规范场
\begin{equation} \label{Amu}
    A_\mu \rightarrow \tilde{A}_\mu = A_\mu + \partial_\mu \lambda
\end{equation}

\begin{example}
    $F_{\mu\nu} \equiv \Dcal _\mu A_\nu - \Dcal_\nu A_\mu = \partial_\mu A_\nu - \partial_\nu A_\mu$ 在 局域规范变换 下不变。

    注意规范变换是仅针对场的变换,$\partial_\mu$ (形式上)不变。由 \refeq{Amu}
    \begin{equation}
        \partial_\mu A_\nu - \partial_\nu A_\mu \rightarrow \partial_\mu \tilde{A}_\nu - \partial_\nu \tilde{A}_\mu = \partial_\mu (A_\nu + \partial_\nu \lambda) - \partial_\nu (A_\mu + \partial_\mu \lambda)
    \end{equation}
\end{example}

\begin{exercise}
    (习题4.1)

    复标量场
    \begin{equation}
        \Lcal = \frac{1}{2} \qty(\partial ^\mu \varphi^* \partial _\mu \varphi - m^2 \varphi^* \varphi)
    \end{equation} 
    局域坐标变换 
    \begin{equation}
        \partial_\mu \varphi  \rightarrow \partial_\mu \qty(e^{-i e \lambda} \varphi) =  e^{-i e \lambda} \qty(\partial_\mu  - i e \partial_\mu \lambda ) \varphi
    \end{equation}
    不会
\end{exercise}




\end{document}


% % \bibliographystyle{apsrev4-2}
% % \bibliography{refs}

