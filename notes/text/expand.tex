% !Tex root = ../main.tex

\section{Expansion}

\subsection{from time to scale factor}

We have two kinds of ``time'' in cosmology since our Universe are expanding. 

If we focus on the \hl{coordinate}, the metric is written as ($c=1$)
\begin{equation}
    \dd s^2 = -\dd t^2 + a^2(t) (\dd x^2 + \dd y^2 + \dd z^2)
\end{equation}
better to write in spherical coordinate system
\begin{equation}
    \dd s^2 = -\dd t^2 + a^2(t) \qty[{\dd{\chi}}^2 + \chi^2 \qty(\dd{\theta}^2 + {\sin}^2{\theta} \dd{\varphi}^2) ]
\end{equation}

By scaling the time axis, we can get the \hl{conformal time}, 
and have an ``appearantly'' Minkovski spacetime
\begin{equation}
    \dd s^2 = a^2(\tau) (-\dd \tau^2 + \dd x^2 + \dd y^2 + \dd z^2)
\end{equation}

The \hl{scale factor} can be solved from the \helper{Friedmann equation}, 
\footnote{also deriveing the Friedmann equations from Einstein field equation is not that trival, see another block in preparation..}
\begin{equation}  \label{eq:Friedmann1}
    {\qty(\frac{\adot}{a})}^2 = \frac{8\pi G}{3} \rho
\end{equation}
with \helper{a model of energy-momentum} contained in the Universe like 
\footnote{Friedmann equation itself has assumptions about the energy-momentum, it's \hl{ideal fluid} which can be parameterized by only $\rho$ and $P$..}
\begin{equation}
    \frac{\rho(t)}{\rho_{\mathrm{cr}}} = \sum_{s=\gamma, \mathrm{m}, \nu, \mathrm{DE}} \Omega_{s} {a(t)}^{-3 (1 + w_{s})} 
\end{equation}
This model assumes a constant of equation of state. 


In our $\Lambda$CDM model (reserve curvature)
\begin{equation} \label{eq:LCDM}
    \frac{\rho(t)}{\rho_{\mathrm{cr}}} = \Omega_{\mathrm{R}} {\qty(\frac{a}{a_0})}^{-4} + \Omega_{\mathrm{M}} {\qty(\frac{a}{a_0})}^{-3} + \Omega_\Lambda + \Omega_{\mathrm{K}} {\qty(\frac{a}{a_0})}^{-2}
\end{equation}
just solve the differencial equation
\begin{equation}
    \dot{a} = H_0 a_0 \sqrt{\Omega_R {\qty(\frac{a}{a_0})}^{-2} + \Omega_M {\qty(\frac{a}{a_0})}^{-1} + \Omega_\Lambda{\qty(\frac{a}{a_0})}^{2} + \Omega_K}
\end{equation}

\begin{equation}
    t(a_1) = \int_{0}^{a_1}  \frac{\dd a}{H_0 a_0 \sqrt{\Omega_R {\qty(\frac{a}{a_0})}^{-2} + \Omega_M {\qty(\frac{a}{a_0})}^{-1} + \Omega_\Lambda {\qty(\frac{a}{a_0})}^{2} + \Omega_K}} 
\end{equation}
It's not linear, so need numerical solution. See code/expansion.ipynb

But it's OK to see some  exceptions.
\begin{block}
    matter donimated:
    \begin{equation}
        \dot{a} = H_0 {(a_0 \Omega_M)}^{\frac{1}{2}}  a^{-\frac{1}{2}} 
    \end{equation}
    then 
    \begin{equation}
        t = \frac{2}{3 H_0 {(a_0 \Omega_M)}^{\frac{1}{2}} } a^{\frac{3}{2}} 
    \end{equation}
    assumed $a(t=0)=0$.
\end{block}

\subsection{to more notations..}

It will be easy to use \hl{redshift} and \hl{Hubble parameter $H(z)$} in the calculations.

\subsubsection{Cosmological Redshift}
Define redshift as
\begin{equation}
    \frac{a}{a_0} = \frac{1}{1+z}
\end{equation}

Redshift $z$ is linked to $a$ and $t$ simply. How about the comoving spacial coordinate $\chi$?

\begin{equation}
    \dd a = - \frac{a_0 \dd z}{{(1+z)}^2}
\end{equation}
The comoving distance from $z_1$ to now ($z_0=0$)
\begin{equation}
    \chi (z_1) = \int_{a_1}^{a_0} \frac{\dd a}{a^2 H(a)} = \int_{z_0}^{z_1} \frac{c \dd{z}}{a_0 H(z)}
\end{equation}
where $z_0=0$ by defination, $c$ is filled to balance the unit. 


\subsubsection{Hubble parameter}
And turn \refeq{Friedmann1} to 
\begin{equation}
    H(a) \equiv \frac{\adot}{a} = H_0 \sqrt{\frac{\rho(a)}{\rho_{\mathrm{cr}}}} \equiv H_0 E(a)
\end{equation}
\refeq{LCDM} turns out 
\begin{gather}
    E(a) = \sqrt{\Omega_{\mathrm{R}} {\qty(\frac{a}{a_0})}^{-4} + \Omega_{\mathrm{M}} {\qty(\frac{a}{a_0})}^{-3} + \Omega_\Lambda + \Omega_{\mathrm{K}} {\qty(\frac{a}{a_0})}^{-2}} \\ 
    E(z) = \sqrt{\Omega_{\mathrm{R}} {\qty(1+z)}^{4} + \Omega_{\mathrm{M}} {\qty(1+z)}^{3} + \Omega_\Lambda + \Omega_{\mathrm{K}} {\qty(1+z)}^{2}} 
\end{gather}

\subsubsection{Conformal Time}

Conformal time identic to the relation between $\chi$ and $t$ in light cone.
\begin{gather}
    \dd \tau^2 = \frac{\dd t^2}{a^2(t)} \\ 
    \dd \tau = \frac{\dd t}{a(t)} = \dd \chi \label{eq:dtau}\\ 
    \tau(t_1) = \int_{t_i}^{t_1} \frac{\dd t}{a(t)}
\end{gather}
The last $=$ in \refeq{dtau} is for the convinence of coding.

\subsection{Hubble sphere}

Hubble sphere is where the Hubble flow velocity\footnote{in the proper length and is not real `velovity'.} equals to the speed of light. 
\begin{equation}
    v_{\mathrm{H}} = \dot{a} \chi  
\end{equation}


Hubble sphere as a function of time should be 
\begin{equation}
    \chi_{\mathrm{H}} =\frac{c}{a H(a)}
\end{equation}

\subsection{Light Cone}

Light cone is defined as $\dd s^2 = 0$, 
the photon starts from $a_1$ and ends at the observer now ($a_0, \chi=0$).
so 
\begin{gather}
    \dd t^2 = a^2(t) \dd \chi^2 \\ 
    \dd t = a(t) \dd \chi \\ 
    \chi_{\rm lc}(a_1) = \int_{t_1}^{t_0} \frac{\dd t}{a(t)} = \int_{a_1}^{a_0} \frac{\dd a }{a \adot} = \int_{a_1}^{a_0} \frac{\dd a }{a^2 H(a)} 
\end{gather} 


\subsection{Event Horizon}

the photon starts from $a_1$ and ends at the observer in the infinity future ($a\rightarrow \infty, \chi=0$). 
\begin{equation}
    \chi_{\rm EH}(a_1) = \int_{t_1}^{\infty} \frac{\dd t}{a(t)} = \int_{a_1}^{\infty} \frac{\dd a }{a^2 H(a)} 
\end{equation}

\subsection{Particle Horizon}

Particle horizon is the trajectory that a photon start from $\chi = 0$ at $t_i$ (in `our model', $t_i = a_i =0$).

\begin{equation}
    \chi_{\rm PH}(a_1) = \int_{t_i}^{t_1} \frac{\dd t}{a(t)} = \int_{a_i}^{a_1} \frac{\dd a }{a \adot} = \int_{a_i}^{a_1} \frac{\dd a }{a^2 H(a)} 
\end{equation}

\subsubsection{optical horizon}

The optical horizon due to recombination $z(t_r) \approx 1100$.
\begin{equation}
    d_{\mathrm{opt}}(t) = a(\tau) \qty(\tau - \tau_r ) = a(t) \int_{t_r}^t \frac{\dd t'}{a(t')}
\end{equation}




\section*{\LaTeX}

% mathbb command is used to convert uppercase and lowercase letters to blackboard-bold in terms of shape, as $\A \bbC$.

\begin{gather}
    \Ham.\\
    \Ham \ket{\psi}
\end{gather}




