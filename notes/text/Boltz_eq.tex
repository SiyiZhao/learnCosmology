% !Tex root = ../main.tex

\section{Boltzmann Equation}

我们讨论 inflation 结束后到 recombination 之前,光子如何产生各向异性/光子的各向异性如何演化。
我们将看到光子和物质之间通过 Thomson scattering 相互作用,将为光子贡献一部分各向异性。(interaction between matter and photon contributes to the **anisotropies of CMB**)

在微观层面,满足 Bosen-Einstein 分布 
\begin{equation}
    f\qty(\eta, \mathbf{x}, E, \mathbf{\hat{p}}) = \frac{1}{e^{E/T}-1}
\end{equation}
说明:  
\begin{itemize}
    \item 注意相空间是 四维时空 加上 四动量。
    \item $\mathbf{\hat{p}}$ 是因为光子动量由能量决定。
    \item thermal dynamic in local coordinate, 
    \item \discuss{thermal dynamic 如何推广到 Minkowski space with perturbation ?这样简单的推广成立吗?——似乎是个还在研究中的问题。}
    \item $f$ is scalar since 相空间的体积元不变,且粒子数不变,$f=$  粒子数/相空间的体积元。
    \item 相体积不变,熵不变,$T$随$E$变。
    \item \discuss{在考虑温度$T$是否改变时聊到了:黑体谱在多普勒红移后还是黑体谱吗?所以聊的结论是啥?}
\end{itemize}


EoM: Boltzmann equation
\begin{equation}
    \frac{\dd f}{\dd \tau} = C[f_a]
\end{equation}
$C[f_a]$ is the interaction with other particles (eg. baryon)

\subsection{0th order}

推出 $\bar{T} \propto \frac{1}{a}$ 

\subsection{1st order}

Step1: 对 $f$ 做 Taylor expansion 
\begin{gather}
    f\qty(\eta, \mathbf{x}, E, \mathbf{\hat{p}}) = \frac{1}{e^{E/T}-1}
\end{gather}

- 算 Boltzmann equation
	- RHS 
		- e-$\gamma$  scattering in e- coordinate (Thomson scattering)
			- $\frac{d \sigma}{d \Omega}$ 
				- 1 给出 monopole 
				- cos theta 给出 quadrupole   
				- 通过这些电子影响光子的分布,给光子 anisotropy ? 
		- 洛伦兹变换 

- fluid description 
- 和 f 比较,得到 CMB temperature multipoles 和 photon fluid variables energy-momentum tensor perturbation multipoles 的关系 

- 以上两件事放在一起,得到 CMB temperature multipoles 
- 和 geodesic 联立,把 epsilon 换成 Bardeen’s potential 
- $\mu = \hat{p} \dot \hat{k}$ 

- 得到 $\Theta$ 的演化方程 
- 和 matter 的 energy-momentum tensor 比较,得到 baryon perturbation , dark matter perturbation 和  baryon perturbation 差一个散射项 
- Einstein field equation 得到 Bardeen’s potential 的演化。
