% Tipo di documento
\documentclass[10pt, a4paper, lualatex]{article}


% Custom package
\usepackage{FileAusiliari/CustomArticle}

% Format di pagina
\usepackage[top=30mm, bottom=30mm, twoside, textwidth=120mm, left=15mm, right=40mm, outer=40mm, marginparsep=8mm, headsep=20pt, marginparwidth=\dimexpr 17.5mm+25pt+5pt]{geometry}


%% ALERT: su Texifier tcbcolback va sostituito con tcbcol@back (PACCHETTO NON AGGIORNATO 29/09/2023)

%%%%%%%%%%%%%%%%%%%%%%%%%%%%%%%%%%%%%%%%%%%%%%%%%%%%%%%%%%%%%%%%%%
%%%%%%%%%%%%%%%%%%%%%%%%%%%%%%%%%%%%%%%%%%%%%%%%%%%%%%%%%%%%%%%%%%

\begin{document}
\title{Titolo}
\author{Mattia Puddu\\mattiapuddu@icloud.com}
\date{\today}
\maketitle
\begin{TitoloIntro}[colbacktitle=red]{\large Abstract}{Introduzione}\end{TitoloIntro}

\section{Expansion}

\subsection{from time to scale factor}

We have two kinds of ``time'' in cosmology since our Universe are expanding. 

If we focus on the \hl{coordinate}, the metric is written as 
\begin{equation}
    d s^2 = dt^2 + a(t) \left(dx^2 + dy^2 + dz^2\right)
\end{equation}

By scaling the time axis, we can get the \hl{conformal time}, which is equalized to the space coordinates
\begin{equation}
    d s^2 = a(t) \left(d\tau^2 + dx^2 + dy^2 + dz^2\right)
\end{equation}

The \hl{scale factor} can be solved from the \helper{Friedmann equation}, 
\begin{equation}
    \left(\frac{\dot{a}}{a}\right)^2 = \frac{8\pi G}{3} \rho
\end{equation}
with \helper{the model of energy-momentum} contained in the Universe 
\begin{equation}
    \frac{\rho(t)}{\rho_{\mathrm{cr}}} = \sum_{s=\gamma, \mathrm{m}, \nu, \mathrm{DE}} \Omega_{s} a(t) ^{-3 (1 + w_{s})} 
\end{equation}
This model assumes a constant 
\begin{equation}
    = \Omega_R \left(\frac{a}{a_0}\right)^{-4} + \Omega_M \left(\frac{a}{a_0}\right)^{-3} + \Omega_\Lambda + \Omega_K \left(\frac{a}{a_0}\right)^{ -2}
\end{equation}
where the second line is for $\Lambda$CDM model.

\note{Friedmann equation itself has assumptions about the energy-momentum, it's \hl{ideal fluid} which can be parameterized by only $\rho$ and $P$..}

\note{also deriveing the Friedmann equations from Einstein field equation is not that trival, see another block in preparation..}


In $\Lambda$CDM model, 
just solve the differencial equation
\begin{equation}
    \dot{a} = H_0 \sqrt{\Omega_R \left(\frac{a}{a_0}\right)^{-2} + \Omega_M \left(\frac{a}{a_0}\right)^{-1} + \Omega_\Lambda\left(\frac{a}{a_0}\right)^{2} + \Omega_K}
\end{equation}
It's not linear, so need numerical solution. 

But it's OK to see some  exceptions.
\begin{Pro}
    matter donimated:
    \begin{equation}
        \dot{a} = H_0 (a_0 \Omega_M)^{\frac{1}{2}}  a^{-\frac{1}{2}} 
    \end{equation}
    then 
    \begin{equation}
        t = \frac{2}{3 H_0 (a_0 \Omega_M)^{\frac{1}{2}} } a^{\frac{3}{2}} 
    \end{equation}
    assumed $a(t=0)=0$.
\end{Pro}

\section*{\LaTeX}

mathbb command is used to convert uppercase and lowercase letters to blackboard-bold in terms of shape, as $\A \bbC$.

\end{document}
